% Created 2022-02-21 Mon 05:32
% Intended LaTeX compiler: pdflatex
\documentclass[11pt]{article}
\usepackage[utf8]{inputenc}
\usepackage[T1]{fontenc}
\usepackage{graphicx}
\usepackage{grffile}
\usepackage{longtable}
\usepackage{wrapfig}
\usepackage{rotating}
\usepackage[normalem]{ulem}
\usepackage{amsmath}
\usepackage{textcomp}
\usepackage{amssymb}
\usepackage{capt-of}
\usepackage{hyperref}
\usepackage{minted}
\usepackage{minted}
\usepackage{tikz}
\usetikzlibrary{positioning}
\usepackage[margin=1in]{geometry}
\usepackage[margin=1in]{pgfplots}
\renewcommand{\Large}{\normalsize}
\author{Amar Panji Senjaya}
\date{\textit{[2022-01-28 Fri 09:15]}}
\title{Discrete-Set}
\hypersetup{
 pdfauthor={Amar Panji Senjaya},
 pdftitle={Discrete-Set},
 pdfkeywords={},
 pdfsubject={},
 pdfcreator={Emacs 27.2 (Org mode 9.4.4)}, 
 pdflang={English}}
\begin{document}

\maketitle
\pgfplotsset{ compat=1.3, compat/path replacement=1.5.1,}\\
\setlength\parindent{0pt}
\newpage

Source And Learning Material:\\
\url{https://www.youtube.com/playlist?list=PLHXZ9OQGMqxersk8fUxiUMSIx0DBqsKZS}\\
\url{https://www.amazon.com/Discrete-Mathematics-Applications-Susanna-Epp/dp/1337694193/ref=sr\_1\_2?keywords=discrete+math\&qid=1645043639\&sr=8-2}\\

\section{What is Set}
\label{sec:orge854395}
A \textbf{set} is a collection of objects. Individual object of set is element \(\in\). Individual object of element that is not in The set is  called not in element denoted with \(\notin\).\\
\subsection{Examples}
\label{sec:org27492f0}
\begin{itemize}
\item A = Employees in a company\\
John \(\in\) A\\
\item B = \{1,3,4,7\}\\
3 \(\in\) A\\
\(\pi\) \(\notin\) B\\
\item Z = Integers\\
3 \(\in\) Z\\
\end{itemize}

\subsection{Order And Repetition Don't Matter}
\label{sec:orgfdfb6fe}
\begin{minted}[]{markdown}
{1, 2, 3, 4, 5, 6, 7} = {2, 5, 4, 3, 1, 7, 6}
			= {2, 2, 2, 2, 5, 4, 3, 1, 1, 7, 6, 6, 6}
\end{minted}

\newpage

\section{Subsets}
\label{sec:org29e07bc}
\begin{itemize}
\item Every Element of A is also in B = A \subseteq B\\
\end{itemize}
\subsubsection{Example 1}
\label{sec:orgc352504}
A = \{1, 3\}\\
B = \{1, 3, 4, 6\}\\
A \subseteq B\\
\subsubsection{Example 2}
\label{sec:org10d0b74}
A = \{1, 8\}\\
B = \{1, 3, 4, 6\}\\
A \nsubseteq B\\
because \(\in\) 8 is not in B\\

\section{Set-Roster Notion}
\label{sec:org8e73dd9}
Replace \ldots{} with continuos clear pattern\\
Example:\\
Positive even integers\\
\{0, 2, 4, 6, \ldots{}\}\\
all even integers\\
\{\ldots{}, -6, -4, -2, 0, 2, 4, 6, \ldots{}\}\\

\section{Set-Builder Notation}
\label{sec:org973438d}
General form: \{ x | P(x) \}\\
x variable\\
"|" such that\\
P(x) Property is true\\
Example even integers\\
= \{ x | x = Twice an integer \}\\
Example square root\\
\{ x | \sqrt{x} \(\in\) \mathbb{Z} \}\\

\newpage

\section{Empty Set}
\label{sec:org0180a82}
Notion: \{\} or \(\emptyset\)\\
\{\(\emptyset\)\}\\
Is \(\emptyset\) \(\subset\) \{1,2,3\}?\\
Recall: A \(\subset\) B means:\\
if x \(\in\) A, then x \(\in\) B\\
This is \emph{vacuously} true!\\

\section{Ordered Pairs (\emph{a}, \emph{b})}
\label{sec:org16c345d}
\begin{itemize}
\item Order Matters\\
\item (a,b) = (c,d) of a=c \& b=d\\
\item a\&b could come from different sets\\
\end{itemize}

\uline{Defn}: The cartesian product A \texttimes{} B is the set of all ordered pairs (a,b) where a \(\in\) A and b \(\in\) B\\

\begin{tikzpicture}
\begin{axis}[legend entries={p},ymin=-2,ymax=2,xmin=-2,xmax=2, axis lines=middle,axis equal,grid=both]
\addplot coordinates{(2,1)};
\end{axis}
\end{tikzpicture}

p = (2,1) \(\in\) \mathbb{R} x \mathbb{R}\\
first component is x axis and second component is y axis\\

Example:\\
\{a,b\} \texttimes{} \{0,1\}\\
A \texttimes{} B = \{(a,1), (a,0), (b,0), (b1)\}\\

\section{Relations}
\label{sec:orgd5ab016}
Ex: a < b\\
compares two integers\\
some pairs have this relationship 2 < 5\\
some pairs don't 5 \nless 2\\

note:\\
h: human\\
d: dog\\
c: cat\\
m: monkey\\
\begin{tikzpicture}[
mydot/.style={
  circle,
  fill,
  inner sep=2pt
},
>=latex,
shorten >= 3pt,
shorten <= 3pt
]
\node[mydot,label={left:h1}] (h1) {}; 
\node[mydot,below=of h1,label={left:h2}] (h2) {}; 
\node[mydot,below=of h2,label={left:h3}] (h3) {}; 
\node[mydot,below=of h3,label={left:h4}] (h4) {}; 

\node[mydot,right=2cm of h1,label={right:d1}] (d1) {}; 
\node[mydot,below=of d1,label={right:c1}] (c1) {}; 
\node[mydot,below=of c1,label={right:d2}] (d2) {}; 
\node[mydot,below=of d2,label={right:m1}] (m1) {}; 

\path[->] (h1) edge (c1);
\path[->] (h2) edge (d1);
\path[->] (h3) edge (d1);
\end{tikzpicture}

\uline{defn}: A Relation R between A and B is a subset of A \texttimes{} B\\

ie ordered pairs\\
(a,b) \(\in\) A \texttimes{} B\\
\section{Functions}
\label{sec:orgcd82465}
ex: f(x) = x\textsuperscript{2}\\

\begin{tikzpicture}
\begin{axis}[legend entries={f},ymin=-2,ymax=2,xmin=-2,xmax=2, axis lines=middle,axis equal,grid=both]
\addplot {(x^2)};
\end{axis}
\end{tikzpicture}

function do something to every input in my domain and produce output for each input\\

domain: set of all possible input\\
range: set of all possible output\\

\uline{Defn}: A function F between A and B\\
is a relation between A and B such that:\\
subset of A \texttimes{} B\\

\begin{enumerate}
\item For every x \(\in\) A there is an element y \(\in\) B such that (x,y) \(\in\) F\\
Which means\\
Fo every input x, there is some ouput y, F(x)=y\\
\item If (x,y) \(\in\) F and (x,z) \(\in\) F then y = z\\
\end{enumerate}


Example: Is this relation?\\
Consider the relation C where (x,y) \(\in\) C if x\textsuperscript{2} + y\textsuperscript{2} = 1. Is this a function?\\
\begin{tikzpicture}
\begin{axis}[legend entries={f},ymin=-2,ymax=2,xmin=-2,xmax=2, axis lines=middle,axis equal,grid=both, extra tick style={grid=major}]
\addplot coordinates{(0,1)};
\draw (axis cs:0,0) circle[radius=1];
\end{axis}
\end{tikzpicture}

This is relation but not a function because there is more than one output asociated with one input\\


\section{Statement}
\label{sec:org05a140f}
A statement is a sentence that is either true or false\\

Examples:\\
p: "5 > 2" = True\\
q: "2 > 5" = False\\
r: "x > 2" = Not a statement because we don't know what x is\\

\subsection{New statement from old}
\label{sec:orgca3a2ce}
\(\neg{}\) p means not p\\
p \(\land\) q means p and q\\
p \(\lor\) q means p or q\\

Example:\\
"My shirt is gray but my shorts are not"\\
p = My shirt is gray\\
q = My shorts are gray\\
p\(\land\) \(\neg{}\) q\\

\subsection{Truth Table for (\(\neg{}\) p) \(\lor\) (\(\neg{}\) q)}
\label{sec:org3c81915}
\begin{center}
\begin{tabular}{lllll}
p & q & \(\neg{}\) p & \(\neg{}\) q & \\
\hline
T & T & F & F & F\\
T & F & F & T & T\\
F & T & T & F & T\\
F & F & T & T & T\\
\end{tabular}
\end{center}

Def: Two statements are logically equivalent if they have the same truth table\\
\begin{center}
\begin{tabular}{lll}
p & \(\neg{}\) p & \(\neg{}\) (\(\neg{}\) p)\\
\hline
T & F & T\\
F & T & F\\
\end{tabular}
\end{center}

Def: A tautology t is a statement that is always true\\
Example:\\
That dog is a mammal\\

t: tautology\\
p: some statement\\
\begin{center}
\begin{tabular}{lll}
t & p & t \(\lor\) p\\
\hline
T & T & T\\
T & F & T\\
\end{tabular}
\end{center}


Def: A contrandiction c is a statement that is always false\\
Example:\\
That dog is a reptile\\

c: contrandiction\\
p: some statement\\
\begin{center}
\begin{tabular}{lll}
c & p & c \(\land\) p\\
\hline
F & T & F\\
F & F & F\\
\end{tabular}
\end{center}

so c \(\land\) p is a contrandiction\\

\newpage
\section{Demorgan's Law \& Logical Equivalent}
\label{sec:org57db3b1}
p: Trefor is a unicorn\\
q: Trefor is a goldfish\\
\(\neg{}\)(p \(\lor\) q) \(\equiv\) (\(\neg{}\) p) \(\land\) (\(\neg{}\) q)?\\
It's NOT the case that Trefor is either a unicorn OR a goldfish.\\
is equivalent to:\\
Trefor is NOT a unicorn AND is NOT a goldfish.\\

\begin{center}
\begin{tabular}{lllllll}
p & q & \(\neg{}\) p & \(\neg{}\) q & p \(\lor\) q & \(\neg{}\) (p \(\land\) q) & (\(\neg{}\) p) \(\land\) (\(\neg{}\) q)\\
\hline
T & T & F & F & T & F & F\\
T & F & F & T & T & F & F\\
F & T & T & F & T & F & F\\
F & F & T & T & F & T & T\\
\end{tabular}
\end{center}

Demorgan's Laws:\\
\(\neg{}\) (p \(\lor\) q) \(\equiv\) (\(\neg{}\) p) \(\land\) (\(\neg{}\) q)\\
\(\neg{}\) (p \(\land\) q) \(\equiv\) (\(\neg{}\) p) \(\lor\) (\(\neg{}\) q)\\

Double Negative:\\
\(\neg{}\)(\(\neg{}\) p) \(\equiv\) p\\

Identity Laws:\\
p \(\lor\) c \(\equiv\) p\\
p \(\land\) t \(\equiv\) p\\

Universal Bound Laws:\\
p \(\lor\) t \(\equiv\) t\\
p \(\land\) c \(\equiv\) c\\

Example:\\
(\(\neg{}\)(p \(\lor\) \(\neg{}\) q)) \(\land\) t\\
via DeMorgan's\\
\(\equiv\) (\(\neg{}\) p \(\land\) \(\neg{}\) (\(\neg{}\) q)) \(\land\) t\\
via Double Negative\\
\(\equiv\) (\(\neg{}\) p \(\land\) q) \(\land\) t\\
Via Identity\\
\(\equiv\) \(\neg{}\) p \(\land\) q\\

So (\(\neg{}\)(p \(\lor\) \(\neg{}\) q)) \(\land\) t \(\equiv\) \(\neg{}\) p \(\land\) q\\

\newpage
\section{Conditional Statement}
\label{sec:orgcfc18be}
Def: p \(\Rightarrow\) q means:\\
"if p is TRUE then q is TRUE"\\
Whenever the hipotesis p is true then the conclution is also true.\\

\begin{center}
\begin{tabular}{lll}
p & q & p \(\Rightarrow\) q\\
\hline
T & T & T\\
T & F & F\\
F & T & T\\
F & F & T\\
\end{tabular}
\end{center}

\begin{center}
\begin{tabular}{ll}
\(\neg{}\) p & \(\neg{}\) p \(\lor\) q\\
\hline
F & T\\
F & F\\
T & T\\
T & T\\
\end{tabular}
\end{center}

p \(\Rightarrow\) q \(\equiv\) \(\neg{}\) p \(\lor\) q\\

Example:\\
If i study hard, then i will pass\\
p = i study hard\\
q = i will pass\\
p \(\Rightarrow\) q\\

Either I don't study hard, or i pass\\
\(\neg{}\) p = I don't study hard\\
q = i pass\\

\subsection{When the hypothesis is false, the statement is \textbf{vacuously true}.}
\label{sec:org35d8fb6}
vacuously true meant the statement is true but true in a sort of unimportant or unintresting or vacuous set.\\

Example:\\
If \uline{Trefor is a unicorn}, then \uline{everyone get's an A}\\
p = Trefor is a unicorn\\
q = everyone get's an A\\
p \(\Rightarrow\) q\\

Example:\\
Either \uline{Trefor is not a unicorn}, or \uline{everyone get's an A}\\
\(\neg{}\) p = Trefor is not a unicorn\\
q =  everyone get's an A\\
\(\neg{}\) p \(\lor\) q\\

\subsection{Negating a conditional}
\label{sec:orga68cf41}
\(\neg{}\) (p \(\Rightarrow\) q) \(\equiv\) \(\neg{}\) (\(\neg{}\) p \(\lor\) q)\\
\(\equiv\) (\(\neg{}\) \(\neg{}\) p \(\land\) \(\neg{}\) q) Demorgan's law\\
\(\equiv\) p \(\land\) \(\neg{}\) q\\

\subsection{Contrapositive of a conditional:}
\label{sec:orgfc6d892}
p \(\Rightarrow\) q \(\equiv\) \(\neg{}\) q \(\Rightarrow\) \(\neg{}\) p\\
p \(\Rightarrow\) q \(\equiv\)  \(\neg{}\) p \(\lor\) q\\
\(\neg{}\) q \(\Rightarrow\) \(\neg{}\) p \(\equiv\) q \(\lor\) \(\neg{}\) p\\


If i study hard, then i will pass p \(\Rightarrow\) q\\
Either I don't study hard, or i pass \(\neg{}\) p \(\lor\) \q\\
If i don't pass, then i didn't study hard \q \(\Rightarrow\) \(\neg{}\) p\\
Either i pass, or I didn't study hard q \(\lor\) \(\neg{}\) p\\

\subsection{The converse and the inverse of a statement}
\label{sec:org3b47c5c}
\subsubsection{The converse statement}
\label{sec:org149b437}
p \(\Rightarrow\) q is the statement q \(\Rightarrow\) p\\
The converse statement is not logically equivalent\\

Example:\\
Not Logically equivalent\\
p \(\Rightarrow\) q = If it's a dog, then it's a mammal = True\\
q \(\Rightarrow\) p = If it's a mammal, then it's a dog = False\\

\subsubsection{The inverse statement}
\label{sec:org2106227}
p \(\Rightarrow\) q is the statement \(\neg{}\) p \(\Rightarrow\) \(\neg{}\) q\\
The inverse statements is not logically equivalent\\

So inverse \(\equiv\) converse\\

Example\\
Not Logically equivalent\\
p \(\Rightarrow\) q = If it's a dog, then it's a mammal = True\\
\(\neg{}\) p \(\Rightarrow\) \(\neg{}\) q = If it's not a dog, then it's not a mammal = False\\

\subsection{Biconditional statement}
\label{sec:orga4a4b93}
The Biconditional p \iff q\\
means that both p \(\Rightarrow\) q and q \(\Rightarrow\) p\\

Example\\
If i study hard, then I will pass = p \(\Rightarrow\) q\\
\uline{AND} if I pass, then I studied hard = q \(\Rightarrow\) p\\

i will pass \textbf{if and only if} i study hard\\
if and only if = \iff\\

\subsection{Valid and Invalid Arguments}
\label{sec:orgfd5c10f}
A valid argument is a list of premises from which the conclusion follows.\\

Example Argument:\\
If \uline{I do the dishes}, then \uline{my wife will be happy with me}.\\
\uline{I do the dishes}.\\
Therefore, \uline{my wife is happy with me}.\\

If \textbf{p}, then \textbf{q}.\\
\textbf{p}.\\
Therefore, \textbf{q}.\\

\subsubsection{Modus Ponens}
\label{sec:orgba51a57}
\textbf{Modus Ponens} is an argument of the form:\\
premise1 = If \textbf{p}, then \textbf{q}.\\
premise2 = \textbf{p}.\\
conclusion = Therefore, \textbf{q}.\\

Variables\\
\begin{center}
\begin{tabular}{ll}
p & q\\
\hline
T & T\\
T & F\\
F & F\\
F & T\\
\end{tabular}
\end{center}

Premises\\
\begin{center}
\begin{tabular}{ll}
p \(\Rightarrow\) q & p\\
\hline
T & T\\
F & T\\
T & F\\
T & F\\
\end{tabular}
\end{center}

Conclusion\\
\begin{center}
\begin{tabular}{l}
q\\
\hline
T\\
X\\
X\\
X\\
\end{tabular}
\end{center}

\subsubsection{Modus Tollens}
\label{sec:orge04826f}
\textbf{Modus Tollens} is an argument of the form:\\
if \textbf{p}, then \textbf{q}.\\
\(\neg{}\) q.\\
Therefore, \(\neg{}\) p.\\

example argument:\\
If i'm POTUS then i'm an American citizen.\\
I'm not an American citizen.\\
Therefore, I'm not the POTUS.\\

p = i'm POTUS\\
q = i'm an American citizen.\\
\(\neg{}\) q = I'm not an American citizen.\\
\(\neg{}\) p = I'm not the POTUS.\\

\subsubsection{Generalization}
\label{sec:org2af5e23}
\textbf{Generalization} is an argument of the form:\\
p.\\
Therefore, p \(\lor\) q.\\

Example:\\
i'm a canadian\\
Therefore, I'm a canadian or i'm a unicorn.\\


\subsubsection{Specialization}
\label{sec:org26807e8}
\textbf{Specialization} is an argument of the form:\\
p \(\land\) q.\\
Therefore, p.\\

Example:\\
I'm a canadian \uline{and} I have a PhD\\
Therefore, I'm a Canadian.\\


\subsubsection{Contradiction}
\label{sec:orge00176f}
\textbf{Contradiction} is an argument of the form:\\
\(\neg{}\) p \(\Rightarrow\) c\\
Therefore, p.\\

Example:\\
If i'm skilled at poker, then I will win.\\
I won money playing poker.\\
Therefore, I'm skilled at poker.\\

p = i'm skilled at poker\\
q = I will win.\\
q = I won money playing poker.\\
p = I'm skilled at poker.\\

p \(\Rightarrow\) q\\
q\\
p\\

This argument is invalid argument because it's use converse statement.\\
We know converse statement is not logically equivalent.\\


\section{Predicates and Quantified Statements}
\label{sec:org54b266c}
Recall: A statement is either TRUE or FALSE\\
\subsection{Predicate}
\label{sec:orgc4bb23c}
a \textbf{predicate} is a sentence depending on variables which becomes a sttemnt upon substituting values in the domain.\\

Example:\\
P(x): x is a factor of 12 with domain \mathbb{Z}\textsuperscript{+}\\

P(6) True\\
P(5) False\\
P(\(\frac{1}{3}\)) Nonsense! \(\frac{1}{3}\) \(\notin\) \mathbb{Z}\textsuperscript{+}\\

\subsection{The Truth set}
\label{sec:org97a4b80}
\textbf{The truth set} of a predicate P(x):\\
\{x \(\in\) \mathbb{D} | P(x)\}\\
i.e All values x in the domain where P(x) is true\\

Example:\\
P(x): x is a factor of 12 with domain \mathbb{Z}\textsuperscript{+}\\
TS = \{1, 2, 3, 4, 5, 6, 12\}\\
\subseteq \mathbb{Z}\textsuperscript{+}\\

\subsection{The Universal Quantifier}
\label{sec:org81fc832}
\textbf{The Universal Quantifier \(\forall\)} means "for all"\\

Main Use "quantifying" predicates\\
\(\forall\) x \(\in\) D, P(x)\\
For all x in the domain, P(x) is true\\

Example:\\
Every dog is a mammal\\
\(\forall\)\\
D = set of dogs ; P(x): X is a mammal\\

\subsection{The existensial quantifier}
\label{sec:org45b8fbc}
\textbf{The existensial quantifier \(\exists\)} means "there exists"\\

Main use "quantifying" predicates\\
\(\exists\) x \(\in\) D, P(x)\\
There exists x in the domain, such that P(x)is true\\

Ex: Some person is the oldest in the world\\
\(\exists\) \(\in\) \{People in the world\}; P(x): X is the oldest\\

statement P: "Roofus is a mammal"\\
predicate P(x): "x is a mammal"\\
statement Q: \(\forall\) x \(\in\) D, P(x): "every dog is a mammal"\\

\subsection{Negating quantifier}
\label{sec:org19e280c}
Negate "\(\forall\) x \(\in\) \mathbb{Z}\textsuperscript{+}, x > 3"\\
\(\exists\) x \(\in\) \mathbb{Z}\textsuperscript{+ },x \ngtr 3\\
\(\exists\) x \(\in\) \mathbb{Z}\textsuperscript{+ },x\(\le\) 3\\
\(\neg{}\) P(x)\\

Negating a universal\\
\(\neg{}\) (\(\forall\) x \mathbb{D}, P(x)) \(\equiv\) x \(\in\) \mathbb{D}, \(\neg{}\) P(x).\\

Example:\\
Negate "Someone in our class it taller than 7 feet"\\
\(\exists\) x \(\in\) D, P(x)\\
D = our class\\
P(x) = x is taller than 7 feet.\\

negation:\\
\(\neg{}\)(\(\exists\) x \(\in\) D, P(x))\\
everyone in our class is shorter than 7 feet\\
\(\forall\) x \(\in\) D, \(\neg{}\) P(x)\\

Negating an existensial\\
\(\neg{}\) (\(\exists\) x \(\in\) D, P(x)) \(\equiv\) \(\forall\) x \(\in\) D, \(\neg{}\) P(x).\\

\subsection{Negating logical statement}
\label{sec:orgd9c841b}
Example\\
Every interger has a larger integer\\
\(\forall\) x \(\in\) \mathbb{Z}, P(x)\\

\(\forall\) x \(\in\) \mathbb{Z}, \(\exists\) y \(\in\) \mathbb{Z}, y > x\\
P(x) = \(\exists\) y \(\in\) \mathbb{Z}, y > x\\

True: choose y = x + 1 \(\in\) \mathbb{Z}\\

Negate: \(\exists\) x \(\in\) \mathbb{Z}, \(\neg{}\) P(x)\\
\(\exists\) x \(\in\) \mathbb{Z}, \(\forall\) y \(\in\) \mathbb{Z}, y \lteq x\\

Example:\\
Some number in D is the largest\\
\(\exists\) x \(\in\) D, P(x)\\
\(\exists\) x \(\in\) D, \(\forall\) y \(\in\) D, X \grteq y\\

Negate: \(\forall\) x \(\in\) D, \(\exists\) y \(\in\) D, x < y\\

\subsection{Universal Conditionals}
\label{sec:orgb196917}
Universal-Conditionals: P(x) \(\Rightarrow\) Q(x)\\
means \(\forall\) x \(\in\) D, P(x) \(\Rightarrow\) Q(x)\\

Example:\\
if x is the POTUS, then x is a US Citizen\\
P(x) = x is the POTUS\\
Q(x) = x is a US Citizen\\
D = people\\

\subsection{Necessary and Sufficient Conditions}
\label{sec:org48d6037}
Square\\
\begin{tikzpicture}
\draw[ultra thick, blue] (0,0) -- (1,0) -- (1,1) -- (0,1) -- (0,0);
\end{tikzpicture}

Rectangle\\
\begin{tikzpicture}
\draw[green!70!dark, ultra thick] (0,0) -- (1.5,0) -- (1.5,1) -- (0,1) -- (0,0);
\end{tikzpicture}

Quadrilateral\\
\begin{tikzpicture}
\draw[red,ultra thick] (0,0) -- (0.9,1) -- (1,-0.3) -- (0.4, -0.5) -- (0,0);
\end{tikzpicture}

All Squares are Rectangles\\
if x is a Square, then x is a Rectangle\\
if A(x), then B(x)\\

A(x) is a sufficient condition for B(x)\\
i.e. "x being a square is sufficient to conclude x is a rectangle"\\

if x is a Rectangle, then x is a Quadrilateral\\
if B(x), then C(x)\\

if x is not a Quadrilateral then x is not a Rectangle\\
if \(\neg{}\) C(x), then \(\neg{}\) B(x)\\

sufficient to have a Square = A(x)\\
if we want a rectangle = B(x)\\
But necessary to have a Quadrilateral C(x)\\
A(x) \(\Rightarrow\) B(x) \(\Rightarrow\) C(x)\\
A(x) is a sufficient condition for B(x)\\
B(x) is a necessary condition for A(x)\\

Being a square is a sufficient condition fo being a rectangle\\
Being a rectangle is a necesary condition for being a square\\

\section{Defining Even \& Odd Integers}
\label{sec:orgc8c7bbb}
Informal Definition: n is an even integer if n can be written as twice an integer.\\
Formal Definition: n is an even integer if \(\exists\) k \(\in\) \mathbb{Z} such that n = 2k\\

Informal Definition: n is an odd integer if n is an integer that is not even.\\
Formal Definition: n is an odd integer if \(\exists\) k \(\in\) \mathbb{Z} such that n = 2k + 1\\

\section{First Proof}
\label{sec:orgd6360c1}
Theorem: an even integer plus an odd integer is another odd integer\\
Proof:\\
Suppose m is even and n is odd.\\
\(\exists\) k\textsubscript{1} \(\in\) Z and \(\exists\) k\textsubscript{2} \(\in\) Z that m = 2k\textsubscript{1} and n = 2k\textsubscript{2} + 1\\

Then, m + n = (2k\textsubscript{1})+(2k\textsubscript{2} + 1)\\
= 2(k\textsubscript{1} + k\textsubscript{2}) + 1\\
Let k\textsubscript{3} = k\textsubscript{1} + k\textsubscript{2}, and note it is an integer.\\

Hence, \(\exists\) k\textsubscript{3} \(\in\) Z so that m + n = 2k\textsubscript{3} + 1\\
Thus m + n is odd.\\

Direct Proofs of: \(\forall\) x \(\in\) D, P(x) \(\Rightarrow\) Q(x)\\
1 State the Assumptions\\
2 Formally Define the assumptions\\
3 Manipulate\\
4 Arrive at Definition of conclusion\\
5 state the conclusion\\
\end{document}