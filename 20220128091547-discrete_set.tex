% Created 2022-01-28 Fri 12:20
% Intended LaTeX compiler: pdflatex
\documentclass[11pt]{article}
\usepackage[utf8]{inputenc}
\usepackage[T1]{fontenc}
\usepackage{graphicx}
\usepackage{grffile}
\usepackage{longtable}
\usepackage{wrapfig}
\usepackage{rotating}
\usepackage[normalem]{ulem}
\usepackage{amsmath}
\usepackage{textcomp}
\usepackage{amssymb}
\usepackage{capt-of}
\usepackage{hyperref}
\usepackage{minted}
\usepackage{minted}
\usepackage[margin=1in]{geometry}
\renewcommand{\Large}{\normalsize}
\author{Amar Panji Senjaya}
\date{\textit{[2022-01-28 Fri 09:15]}}
\title{Discrete-Set}
\hypersetup{
 pdfauthor={Amar Panji Senjaya},
 pdftitle={Discrete-Set},
 pdfkeywords={},
 pdfsubject={},
 pdfcreator={Emacs 27.2 (Org mode 9.4.4)}, 
 pdflang={English}}
\begin{document}

\maketitle
\setlength\parindent{0pt}
\newpage

\section{What is Set}
\label{sec:orge84a5b3}
A \textbf{set} is a collection of objects. Individual object of set is element \(\in\). Individual object of element that is not in The set is  called not in element denoted with \(\notin\).\\
\subsection{Examples}
\label{sec:org885b229}
\begin{itemize}
\item A = Employees in a company\\
John \(\in\) A\\
\item B = \{1,3,4,7\}\\
3 \(\in\) A\\
\(\pi\) \(\notin\) B\\
\item Z = Integers\\
3 \(\in\) Z\\
\end{itemize}

\subsection{Order And Repetition Don't Matter}
\label{sec:org5867a76}
\begin{minted}[]{markdown}
{1, 2, 3, 4, 5, 6, 7} = {2, 5, 4, 3, 1, 7, 6}
			= {2, 2, 2, 2, 5, 4, 3, 1, 1, 7, 6, 6, 6}
\end{minted}

\newpage

\subsection{Subsets}
\label{sec:org20d823d}
\begin{itemize}
\item Every Element of A is also in B = A \subseteq B\\
\end{itemize}
\subsubsection{Example 1}
\label{sec:orgf7fbae4}
A = \{1, 3\}\\
B = \{1, 3, 4, 6\}\\
A \subseteq B\\
\subsubsection{Example 2}
\label{sec:orga5603c4}
A = \{1, 8\}\\
B = \{1, 3, 4, 6\}\\
A \nsubseteq B\\
because \(\in\) 8 is not in B\\

\subsection{Set-Roster Notion}
\label{sec:orgbd47beb}
Replace \ldots{} with continuos clear pattern\\
Example:\\
Positive even integers\\
\{0, 2, 4, 6, \ldots{}\}\\
all even integers\\
\{\ldots{}, -6, -4, -2, 0, 2, 4, 6, \ldots{}\}\\

\subsection{Set-Builder Notation}
\label{sec:orgbc0563b}
General form: \{ x | P(x) \}\\
x variable\\
"|" such that\\
P(x) Property is true\\
Example even integers\\
= \{ x | x = Twice an integer \}\\
Example square root\\
\{ x | \sqrt{x} \(\in\) \mathbb{Z} \}\\
\end{document}